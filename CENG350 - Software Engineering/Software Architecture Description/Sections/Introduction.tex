\pagenumbering{arabic}

\chapter{Introduction} \label{introduction}
This document is the Software Architecture Description (SAD) of an open source system called FarmBot \cite{FarmBotWhitepaper}. The system has a website \cite{FarmBotWebsite} and a GitHub repository \cite{FarmBotGitHub}.

\section{Purpose and objectives of FarmBot}
FarmBot harmonizes the efficiencies of monocrop and polycrop systems, aiming to create a sustainable and productive agricultural model. While monocropping makes use of machinery and reduces labor through uniform crops, it often leads to higher resource use and is harmful to the environment. Polycropping's diverse ecosystem reduces these inputs but typically lacks machinery support, requiring more labor. FarmBot's scalable technology bridges this gap, offering machine efficiency to diverse crops with minimal labor, similar to monocropping. The core purpose of FarmBot is to cultivate farms that combine abundance and resistivity with sustainability and efficiency. Objectives include:
\begin{itemize}
    \item Enhancing the biological efficiency of polycrops while maintaining mechanical efficiency.
    \item Reducing dependency on external inputs like fertilizers and pesticides.
    \item Minimizing environmental impacts such as soil depletion and water pollution.
    \item Promoting sustainable farming through technological innovation.
    \item Achieving large-scale operations with minimal human labor.
\end{itemize}

\section{Scope}
FarmBot aims to modernize agriculture by automation, and it helps to apply both individual and commercial farming operations requiring efficiency, sustainability, and scalability. The project will deliver FarmBot's integrated software package, an open-source cloud-based Software as a Service (SaaS) solution. \\\\
FarmBot offers a bunch of key benefits aimed at transforming modern agriculture: it boosts crop yields, minimizes waste of resources, and improves operational efficiency. This software is designed to generalize precision farming, making it accessible to a wider audience by harnessing technology for sustainable agricultural practices. The goals of the FarmBot project are to make advanced farming technology more universally applicable, support data-driven agricultural methods, and inspire society to adopt cutting-edge agricultural systems. The development and features of the FarmBot software are rigorously crafted to support these objectives, which ensure a direct contribution to enhancing agricultural productivity and sustainability through technological advancement. Some components of the software are:
\begin{itemize}
    \item A user-friendly web frontend which enable users to design and plan farm layouts.
    \item Decision support system which includes data mapping and analysis tools that helps the user to store, access, and analyze farming data, thus, utilize analytics for optimizing farming operations.
    \item User, farm, and equipment profile management systems. By means of user profiles, work and data of the user can be saved and later accessed. Also, user profiles provide authentication against the hacking purposes. Farm profiles include the data associated with the farm including plant layouts, scheduled operations, statistics, and history of the farm. Via the equipment profiles, information about equipments is stored and easily modified as equipment is upgraded or become inactive for maintenance.
    \item Microcontroller software for direct hardware interaction that helps the user to control FarmBot hardware.
    \item Open data repositories for community-driven knowledge sharing.
\end{itemize}

\section{Stakeholders and their concerns}
The FarmBot project comprises a large spectrum of stakeholders, each with varying levels of technical expertise and operational needs. The stakeholders include:
\begin{itemize}
    \item \textbf{Hobbyist Gardeners:} Their main concern is the ease of use and reliability of FarmBot to cultivate their home gardens. They require a system that is intuitive and doesn't necessitate advanced technical knowledge.
    \item \textbf{Professional Farmers:} They prioritize FarmBot's efficiency and scalability to ensure it integrates smoothly into larger-scale farming operations, often focusing on the system's capability to increase crop yield and reduce manual labor.
    \item \textbf{Educators:} They utilize FarmBot as an educational device, hence they are interested in the system's functionality and its potential as a teaching aid to demonstrate modern farming techniques.
    \item \textbf{Researchers:} They utilize FarmBot for experimental farming methods and crop studies. Their concerns include the adaptability of the system to different research settings and the accuracy of data collection for sustainable agriculture practices.
    \item \textbf{Software Developers:} A group primarily concerned with the system's programmability and open-source nature, ensuring that enhancements and community contributions can be made effectively.
    \item \textbf{System Administrators:} Focus on maintaining the FarmBot system, particularly its operational integrity and data security. They need comprehensive control over the software and hardware for updates and troubleshooting.
\end{itemize}
Each stakeholder group contributes to the FarmBot ecosystem, driving the project towards a future where technology and agriculture unite to create efficient and sustainable farming practices.
