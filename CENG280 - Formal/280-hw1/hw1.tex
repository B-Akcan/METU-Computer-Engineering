\documentclass{article}
\usepackage{amsmath}




\begin{document}
\section*{Student Info}
Name: Batuhan Akçan\\
Number: 2580181
\section*{Question 1}
\subsection*{a)} False. Because for example $\pi$ can not be represented over $\Sigma$.
\subsection*{b)} False. Only countably many languages can be represented.
\subsection*{c)} True. If you take 0 $a$'s from the first $a^*$, 2 $b$'s from the first $b^*$, 1 $a$ from the second $a^*$, and 0 $b$'s from the second $b^*$; then $bba$ will be generated.
\subsection*{d)} False. We can form the string $aab$ with the given regular expression, which does not have $ab$ as prefix.
\section*{Question 2}
\subsection*{a)}
$K =\{q_0, q_1, q_2, q_3, q_4\}$\vspace{0.2cm}\\
$\Sigma = \{a, b\}$\vspace{0.2cm}\\
$s = q_0$\vspace{0.2cm}\\
$F = \{q_0,q_1,q_2,q_3\}$\vspace{0.2cm}\\
$\delta(q_0,a) = q_1, \; \delta(q_0,b) = q_2, \; \delta(q_1,a) = q_1, \; \delta(q_1,b)=q_3, \; \delta(q_2,a)=q_1, \\ \delta(q_2,b)=q_2, \; \delta(q_3,a)=q_4, \; \delta(q_3,b)=q_0, \; \delta(q_4,a)=q_4, \; \delta(q_4,b)=q_4$
\subsection*{b)}
$(q_0, abbaabab) \vdash_M (q_1, bbaabab) \vdash_M (q_3, baabab) \vdash_M (q_0, aabab) \vdash_M (q_1, abab) \vdash_M (q_1, bab) \vdash_M (q_3,ab) \vdash_M (q_4,b) \vdash_M (q_4,e)$\vspace{0.3cm}\\
The DFA will not accept the input because it ends at $q_4$, which is the trap state (not a final state). In other words, $abbaabab \notin L(M).$
\section*{Question 3}
\subsection*{a)}
$E(q_0) = \{q_0, q_2\}$\vspace{0.1cm}\\
$E(q_1) = \{q_1\}$\vspace{0.1cm}\\
$E(q_2) = \{q_2\}$\vspace{0.1cm}\\
$E(q_3) = \{q_0, q_2, q_3\}$\vspace{0.1cm}\\
$E(q_4) = \{q_0, q_2, q_3, q_4\}$
\subsection*{b)}
Step 1: True.\vspace{0.1cm}\\
Step 2: True.\vspace{0.1cm}\\
Step 3: False. $s' = E(s) \;$ which definitely has the element $s$, but it can also have other elements.\vspace{0.1cm}\\
Step 4: False. $F'$ consists of the elements of $K'$. Those elements of $K'$ definitely have at least one state $\; q \in F \;$, but they can have other states too. The other states do not have to be in $F$.\vspace{0.1cm}\\
Step 5: False. The transition function $\delta$ takes two inputs: an element $Q$ of $K'$ and an element $a$ of $\Sigma'$.  The function returns the set which is the union of $E(p_i)$'s for $p_i \in K$, for which there exists a $q \in Q$ and $(q,a,p_i) \in \Delta$. 
























\end{document}